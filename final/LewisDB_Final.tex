\documentclass[]{article}
\usepackage{lmodern}
\usepackage{amssymb,amsmath}
\usepackage{ifxetex,ifluatex}
\usepackage{fixltx2e} % provides \textsubscript
\ifnum 0\ifxetex 1\fi\ifluatex 1\fi=0 % if pdftex
  \usepackage[T1]{fontenc}
  \usepackage[utf8]{inputenc}
\else % if luatex or xelatex
  \ifxetex
    \usepackage{mathspec}
  \else
    \usepackage{fontspec}
  \fi
  \defaultfontfeatures{Ligatures=TeX,Scale=MatchLowercase}
\fi
% use upquote if available, for straight quotes in verbatim environments
\IfFileExists{upquote.sty}{\usepackage{upquote}}{}
% use microtype if available
\IfFileExists{microtype.sty}{%
\usepackage{microtype}
\UseMicrotypeSet[protrusion]{basicmath} % disable protrusion for tt fonts
}{}
\usepackage[margin=1in]{geometry}
\usepackage{hyperref}
\hypersetup{unicode=true,
            pdfborder={0 0 0},
            breaklinks=true}
\urlstyle{same}  % don't use monospace font for urls
\usepackage{color}
\usepackage{fancyvrb}
\newcommand{\VerbBar}{|}
\newcommand{\VERB}{\Verb[commandchars=\\\{\}]}
\DefineVerbatimEnvironment{Highlighting}{Verbatim}{commandchars=\\\{\}}
% Add ',fontsize=\small' for more characters per line
\usepackage{framed}
\definecolor{shadecolor}{RGB}{248,248,248}
\newenvironment{Shaded}{\begin{snugshade}}{\end{snugshade}}
\newcommand{\KeywordTok}[1]{\textcolor[rgb]{0.13,0.29,0.53}{\textbf{#1}}}
\newcommand{\DataTypeTok}[1]{\textcolor[rgb]{0.13,0.29,0.53}{#1}}
\newcommand{\DecValTok}[1]{\textcolor[rgb]{0.00,0.00,0.81}{#1}}
\newcommand{\BaseNTok}[1]{\textcolor[rgb]{0.00,0.00,0.81}{#1}}
\newcommand{\FloatTok}[1]{\textcolor[rgb]{0.00,0.00,0.81}{#1}}
\newcommand{\ConstantTok}[1]{\textcolor[rgb]{0.00,0.00,0.00}{#1}}
\newcommand{\CharTok}[1]{\textcolor[rgb]{0.31,0.60,0.02}{#1}}
\newcommand{\SpecialCharTok}[1]{\textcolor[rgb]{0.00,0.00,0.00}{#1}}
\newcommand{\StringTok}[1]{\textcolor[rgb]{0.31,0.60,0.02}{#1}}
\newcommand{\VerbatimStringTok}[1]{\textcolor[rgb]{0.31,0.60,0.02}{#1}}
\newcommand{\SpecialStringTok}[1]{\textcolor[rgb]{0.31,0.60,0.02}{#1}}
\newcommand{\ImportTok}[1]{#1}
\newcommand{\CommentTok}[1]{\textcolor[rgb]{0.56,0.35,0.01}{\textit{#1}}}
\newcommand{\DocumentationTok}[1]{\textcolor[rgb]{0.56,0.35,0.01}{\textbf{\textit{#1}}}}
\newcommand{\AnnotationTok}[1]{\textcolor[rgb]{0.56,0.35,0.01}{\textbf{\textit{#1}}}}
\newcommand{\CommentVarTok}[1]{\textcolor[rgb]{0.56,0.35,0.01}{\textbf{\textit{#1}}}}
\newcommand{\OtherTok}[1]{\textcolor[rgb]{0.56,0.35,0.01}{#1}}
\newcommand{\FunctionTok}[1]{\textcolor[rgb]{0.00,0.00,0.00}{#1}}
\newcommand{\VariableTok}[1]{\textcolor[rgb]{0.00,0.00,0.00}{#1}}
\newcommand{\ControlFlowTok}[1]{\textcolor[rgb]{0.13,0.29,0.53}{\textbf{#1}}}
\newcommand{\OperatorTok}[1]{\textcolor[rgb]{0.81,0.36,0.00}{\textbf{#1}}}
\newcommand{\BuiltInTok}[1]{#1}
\newcommand{\ExtensionTok}[1]{#1}
\newcommand{\PreprocessorTok}[1]{\textcolor[rgb]{0.56,0.35,0.01}{\textit{#1}}}
\newcommand{\AttributeTok}[1]{\textcolor[rgb]{0.77,0.63,0.00}{#1}}
\newcommand{\RegionMarkerTok}[1]{#1}
\newcommand{\InformationTok}[1]{\textcolor[rgb]{0.56,0.35,0.01}{\textbf{\textit{#1}}}}
\newcommand{\WarningTok}[1]{\textcolor[rgb]{0.56,0.35,0.01}{\textbf{\textit{#1}}}}
\newcommand{\AlertTok}[1]{\textcolor[rgb]{0.94,0.16,0.16}{#1}}
\newcommand{\ErrorTok}[1]{\textcolor[rgb]{0.64,0.00,0.00}{\textbf{#1}}}
\newcommand{\NormalTok}[1]{#1}
\usepackage{graphicx,grffile}
\makeatletter
\def\maxwidth{\ifdim\Gin@nat@width>\linewidth\linewidth\else\Gin@nat@width\fi}
\def\maxheight{\ifdim\Gin@nat@height>\textheight\textheight\else\Gin@nat@height\fi}
\makeatother
% Scale images if necessary, so that they will not overflow the page
% margins by default, and it is still possible to overwrite the defaults
% using explicit options in \includegraphics[width, height, ...]{}
\setkeys{Gin}{width=\maxwidth,height=\maxheight,keepaspectratio}
\IfFileExists{parskip.sty}{%
\usepackage{parskip}
}{% else
\setlength{\parindent}{0pt}
\setlength{\parskip}{6pt plus 2pt minus 1pt}
}
\setlength{\emergencystretch}{3em}  % prevent overfull lines
\providecommand{\tightlist}{%
  \setlength{\itemsep}{0pt}\setlength{\parskip}{0pt}}
\setcounter{secnumdepth}{0}
% Redefines (sub)paragraphs to behave more like sections
\ifx\paragraph\undefined\else
\let\oldparagraph\paragraph
\renewcommand{\paragraph}[1]{\oldparagraph{#1}\mbox{}}
\fi
\ifx\subparagraph\undefined\else
\let\oldsubparagraph\subparagraph
\renewcommand{\subparagraph}[1]{\oldsubparagraph{#1}\mbox{}}
\fi

%%% Use protect on footnotes to avoid problems with footnotes in titles
\let\rmarkdownfootnote\footnote%
\def\footnote{\protect\rmarkdownfootnote}

%%% Change title format to be more compact
\usepackage{titling}

% Create subtitle command for use in maketitle
\newcommand{\subtitle}[1]{
  \posttitle{
    \begin{center}\large#1\end{center}
    }
}

\setlength{\droptitle}{-2em}

  \title{}
    \pretitle{\vspace{\droptitle}}
  \posttitle{}
    \author{}
    \preauthor{}\postauthor{}
    \date{}
    \predate{}\postdate{}
  

\begin{document}

%\onehalfspacing
\pagenumbering{gobble}

%\begin{titlepage}
\begin{center}
\LARGE{\textbf{Identification cell type marker genes of the brain and their use in estimation of cell type proportions}}\\
\vspace*{2\baselineskip}
\Large{\textbf{Thesis Proposal for Doctor of Philosophy(PhD) Degree}}\\
\normalsize{UBC bioinformatics Graduate Program}\\
\vspace*{2\baselineskip}
\Large{Ogan Mancarci, B.Sc}\\
\vspace*{3\baselineskip}
\Large{\textbf{Thesis Supervisor}}\\
Dr. Paul Pavlidis\\
\vspace*{2\baselineskip}
\Large{\textbf{Committee Members}}\\
Dr. Clare Beasley\\
Dr. Shernaz Bamji\\
Dr. Sara Mostafavi\\
\vspace*{1\baselineskip}
\Large{\textbf{Chair}}\\
Dr. Ryan Brinkman\\
\vspace*{1\baselineskip}
\Large{\textbf{Examination Date}}\\
June 19, 2015
\end{center}
% \end{titlepage}

\doublespacing

\hypersetup{linkcolor = black}
\newpage
\pagenumbering{roman}
\tableofcontents
\addcontentsline{toc}{section}{\contentsname}

\newpage

\pagenumbering{arabic}
\hypersetup{linkcolor = blue}

\section{1 Introduction}\label{introduction}

As a once long term resident of the great state of New York. The author
aims to learn more about its most populious city through this project.
The City of New York, New York has generated an Open Data iniative in
order to be fair and open with its consituents. As the author may be
taking jobs in the city as early as 2020, there must be a way to
determine the dangerous areas of the city from the crime data. In order
to do this the student has referenced the New York Police Department
(NYPD) public facing records for shooting crime violations during the
year 2018. This sourced file was provided free to all on the internet
and has used the open source program R to analyze this data and present
it in a way that anyone can see and understand. This file contains
115,326 data points that will analyzed.

\begin{center}\rule{0.5\linewidth}{\linethickness}\end{center}

\section{2 Business Questions}\label{business-questions}

This project will attempt to answer the following two questions:

\begin{itemize}
\tightlist
\item
  Does time of day and location have a direct corilation to a shooting
  event?
\item
  Has any part of the city consistantly been aprt of the
\end{itemize}

Reason for this understanding

\begin{center}\rule{0.5\linewidth}{\linethickness}\end{center}

\section{3 Data Acquisition, Cleansing Transformation,
Munging}\label{data-acquisition-cleansing-transformation-munging}

\subsection{3.1 Problem Definition}\label{problem-definition}

\subsection{3.2 Data Acquisition}\label{data-acquisition}

The data for this report was gathered from
\href{https://data.cityofnewyork.us/Public-Safety/NYPD-Shooting-Incident-Data-Historic-/833y-fsy8}{New
York Cities Open Data Project}. The initial setup for this file is a
comma seperated values document with a total of 6,407 rows with 18
attributes for a total of 115,326 data points.

\begin{Shaded}
\begin{Highlighting}[]
\NormalTok{urlToImport <-}\StringTok{ "https://data.cityofnewyork.us/api/views/833y-fsy8/rows.csv?accessType=DOWNLOAD"}
\NormalTok{rawCSV <-}\StringTok{ }\KeywordTok{data.frame}\NormalTok{((}\KeywordTok{read.csv}\NormalTok{(urlToImport)))}
\end{Highlighting}
\end{Shaded}

\subsection{3.3 Data Clensing Process}\label{data-clensing-process}

\subsection{3.4 Data Dictionary}\label{data-dictionary}

\begin{center}\rule{0.5\linewidth}{\linethickness}\end{center}

\section{4 Descriptive Statistics}\label{descriptive-statistics}

\subsection{4.1 Summary Statistics}\label{summary-statistics}

\subsection{4.2 Data Structure}\label{data-structure}

\subsection{4.3 Re-shaping the Data}\label{re-shaping-the-data}

\subsection{4.4 Graphs, Charts, Tables}\label{graphs-charts-tables}

\section{5 Modeling Techniques}\label{modeling-techniques}

\section{6 Data Interpretation}\label{data-interpretation}

\section{7 Summary}\label{summary}

Actionable ideas or insights

\section{8 Appendix}\label{appendix}

\subsection{A.1 R Code}\label{a.1-r-code}

\begin{Shaded}
\begin{Highlighting}[]
\NormalTok{#### Generate a Function to Ensure a Package is Installed }\AlertTok{###}
\NormalTok{EnsurePackage<-}\ControlFlowTok{function}\NormalTok{(x)\{}
\NormalTok{  x<-}\KeywordTok{as.character}\NormalTok{(x)}
  \ControlFlowTok{if}\NormalTok{ (}\OperatorTok{!}\KeywordTok{require}\NormalTok{(x,}\DataTypeTok{character.only=}\OtherTok{TRUE}\NormalTok{))\{}
    \KeywordTok{install.packages}\NormalTok{(}\DataTypeTok{pkgs=}\NormalTok{x, }\DataTypeTok{repos=}\StringTok{"http://cran.r-project.org"}\NormalTok{)}
    \KeywordTok{require}\NormalTok{(x, }\DataTypeTok{character.only=}\OtherTok{TRUE}\NormalTok{)}
\NormalTok{  \}}
\NormalTok{\}}
\NormalTok{###########Import All Required Packages#######################}
\KeywordTok{EnsurePackage}\NormalTok{(}\StringTok{"ggplot2"}\NormalTok{)}
\KeywordTok{EnsurePackage}\NormalTok{(}\StringTok{"ggmap"}\NormalTok{)}
\KeywordTok{EnsurePackage}\NormalTok{(}\StringTok{"gridExtra"}\NormalTok{)}
\KeywordTok{EnsurePackage}\NormalTok{(}\StringTok{"maptools"}\NormalTok{)}
\KeywordTok{EnsurePackage}\NormalTok{(}\StringTok{"RJSONIO"}\NormalTok{)}

\NormalTok{##############Gather and Load the Data########################}
\NormalTok{urlToImport <-}\StringTok{ "https://data.cityofnewyork.us/api/views/833y-fsy8/rows.csv?accessType=DOWNLOAD"}
\NormalTok{rawCSV <-}\StringTok{ }\KeywordTok{data.frame}\NormalTok{((}\KeywordTok{read.csv}\NormalTok{(urlToImport)))}

\NormalTok{########Create the Data Dictionary############################}
\NormalTok{varName<-}\KeywordTok{c}\NormalTok{(}\StringTok{"list of names"}\NormalTok{)}
\NormalTok{varType<-}\KeywordTok{c}\NormalTok{(}\StringTok{"list of types"}\NormalTok{)}
\NormalTok{varDesc<-}\KeywordTok{c}\NormalTok{(}\StringTok{"list of descriptions"}\NormalTok{)}
\NormalTok{dataDict<-}\KeywordTok{data.frame}\NormalTok{(varName,varType,varDesc)}
\KeywordTok{colnames}\NormalTok{(dataDict)<-}\KeywordTok{c}\NormalTok{(}\StringTok{"Variable Name"}\NormalTok{, }\StringTok{"Variable Type"}\NormalTok{, }\StringTok{"Variable Description"}\NormalTok{)}
\KeywordTok{grid.table}\NormalTok{(dataDict)}

\NormalTok{########Create the map to plot on#############################}
\end{Highlighting}
\end{Shaded}

\subsection{A.2 Notes}\label{a.2-notes}


\end{document}
